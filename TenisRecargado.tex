\documentclass[a4paper,12pt]{article}
\usepackage[left=2.5cm,top=5.5cm,right=1.5cm,bottom=2.5cm,headheight=100pt]{geometry}
\usepackage[spanish]{babel}
\usepackage[utf8]{inputenc} 
\usepackage{multicol}
\usepackage{framed, color}
\usepackage{float}
\usepackage{ragged2e} 
\usepackage{graphicx}
\usepackage{fancyhdr}
\usepackage{amsmath}
\usepackage{hyperref}

\setlength{\columnsep}{0.5cm}
 
\begin{document}
\pagestyle{fancy}
\lhead{\raggedright\textbf{\large{Programación Avanzada (1113) - Programación III (617) UNLaM \\\textbf{\large{Instancia:}}\\\textbf{\large{Fecha:}}\\\textbf{\large{Alumnos:}}\\}}}
\rhead{\begin{picture}(0,0) \put(-56.7,10){\includegraphics[width=19mm]{unlam}} \end{picture}}
\lfoot{Programación Avanzada (1113)\\Programación III (617)}
\rfoot{UNLaM}

%\fancyfoot{}
\begin{multicols}{2}[\center\textbf{\large{Tenis Recargado\\\small{Autor: Juan Cruz Piñero - Universidad Nacional del Comahue\\ 
Source: Torneo Argentino de Programación ACM-ICPC 2015 \\}}}]

\author{}

\justify
\textbf{Descripción del problema:}\\
CompuTenis es una disciplina en la cual compiten dos jugadores en un partido, a los que denominaremos A y B. Gana el partido aquel jugador que gane primero S sets, cada uno de los cuales está compuesto por uno o mas juegos. En cada set se disputan tantos juegos como hagan falta para que alguno de los dos jugadores venza en al menos J juegos, con una diferencia de al menos D juegos ganados más que su oponente. Aquel jugador que cumpla con ambas condiciones es entonces el ganador del set correspondiente. La Asociación de Clubes Modernos (ACM) encontró recientemente un registro de partidos de CompuTenis prehistóricos. Cada registro consiste en una cadena compuesta por N caracteres ‘A’ o ‘B’, indicando el jugador que ganó cada uno de los N juegos que tuvo el partido, en el orden en el que fueron sucediendo. Ahora la ACM quiere saber, para cada registro, cuál fue el resultado del partido 
.\\

\textbf{Datos de entrada:}\\
Se recibe un archivo prehistorico.in con el siguiente formato:\\
\begin{itemize}
\item La primera línea contiene cuatro enteros N, S, J y D. El valor N representa la cantidad de juegos presentes en el registro a analizar ($1 \leq N \leq 105$). El valor S indica la cantidad de sets que es necesario ganar para ganar un partido ($1 \leq S \leq 10$). El valor J es la cantidad mínima de juegos que es necesario ganar para ganar un set, mientras que el valor D indica que un jugador debe ganar al menos esa cantidad de juegos más que su oponente para ganar el set ($1 \leq D \leq J \leq 100$). 
\item La segunda línea contiene una cadena compuesta por N caracteres ‘A’ o ‘B’. El i-ésimo caracter de la cadena indica qué jugador ganó el i-ésimo juego disputado en el partido. La cadena de la entrada representa un
registro válido de un partido completo.

\end{itemize}

\textbf{Datos de salida:} \\
Se debe generar un archivo resultado.out conteniendo una línea con dos enteros que representan la cantidad de sets ganados por el jugador A y por el jugador B, respectivamente.\\




\textbf{Ejemplo 1:} \\
Si el archivo \textbf{prehistorico.in} fuera\\

\fbox{\parbox[b]{\linewidth}{10 5 2 1\\AAAAAAAAAA}}
\\

El archivo \textbf{resultado.out} será:\\

\fbox{\parbox[b]{\linewidth}{5 0 }}\\

\textbf{Ejemplo 2} \\
Si el archivo \textbf{prehistorico.in} fuera\\

\fbox{\parbox[b]{\linewidth}{21 3 3 2\\AABABBBABBBABABABBABB}}
\\

El archivo \textbf{resultado.out} será:\\

\fbox{\parbox[b]{\linewidth}{1 3}}

\end{multicols}
\end{document}
