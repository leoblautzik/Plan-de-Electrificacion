\documentclass[a4paper,12pt]{article}
\usepackage[left=2.5cm,top=5.5cm,right=1.5cm,bottom=2.5cm,headheight=100pt]{geometry}
\usepackage[spanish]{babel}
\usepackage[utf8]{inputenc} 
\usepackage{multicol}
\usepackage{framed, color}
\usepackage{float}
\usepackage{ragged2e} 
\usepackage{graphicx}
\usepackage{fancyhdr}
\usepackage{amsmath}
\usepackage{hyperref}

\setlength{\columnsep}{0.5cm}
 
\begin{document}
\pagestyle{fancy}
\lhead{\raggedright\textbf{\large{Programación Avanzada (1113) - Programación III (617) UNLaM \\\textbf{\large{Instancia:}}\\\textbf{\large{Fecha:}}\\\textbf{\large{Alumnos:}}\\}}}
\rhead{\begin{picture}(0,0) \put(-56.7,10){\includegraphics[width=19mm]{unlam}} \end{picture}}
\lfoot{Programación Avanzada (1113)\\Programación III (617)}
\rfoot{UNLaM}

%\fancyfoot{}
\begin{multicols}{2}[\center\textbf{\large{Plan de Electrificación\\\small{\url{http://acm.timus.ru/problem.aspx?space=1&num=1982}\\Problem Author: Mikhail Rubinchik\\ 
Problem Source: Open Ural FU Championship 2013\\}}}]

\author{}

\justify
En un lejano país hay $n$ ciudades. El gobierno ha decidido electrificar todas estas ciudades. Al principio, se construyeron centrales en $k$ ciudades diferentes. Las otras ciudades deben estar conectadas con las centrales eléctricas a través de líneas eléctricas. Para cualquier par de ciudades, es posible construir una línea de transporte de electricidad con un costo de $c_{ij}$ rublos. El país está en crisis después de una guerra civil, por lo que el gobierno decidió construir sólo unas pocas líneas eléctricas. Por supuesto de cada ciudad debe haber un camino a lo largo de las líneas a alguna ciudad con una central eléctrica. Encuentre el costo mínimo posible para construir todas las líneas eléctricas necesarias.
\justify
\textbf{Datos de entrada:}
La primer linea contiene dos enteros $n$ y $k$ $(k \leq n \leq 100)$.
La segunda linea contiene $k$ diferentes enteros que representan los números de ciudades que tienen centrales eléctricas. 
Las siguientes $n$ lineas contienen una tabla de $n \times n$ enteros $c_{ij}$, donde: $c_{ij} >0$ si $i \not = j$ y $c_{ii} = 0$.
\justify
\textbf{Datos de salida:}
Una linea indicando el costo de electrificar las ciudades.
A continuación, $m$ lineas, donde cada una representa las lineas que componen el tendido resultante. 
%\begin{multicols}{2}

\justify
\textbf{Ejemplo:} 
Si su programa leyera el archivo ciudades.in siguiente:\\

%\begin {centering}
\noindent
\fbox{\parbox[b][\height][t]{0.20\textwidth}{
4 2\\
1 4\\
0 2 4 3\\
2 0 5 2\\
4 5 0 1\\
3 2 1 0\\
} }

%\end {centering}

\justify Debería escribir el archivo tendido.out siguiente:\\

%\begin {centering}
\noindent
\fbox{\parbox[b][\height][t]{0.20\textwidth}{
3\\
2 4\\
3 4\\}}

%\end {centering}

%\end{multicols}
\end{multicols}
%\url{http://acm.timus.ru/problem.aspx?space=1&num=1982}
% \begin{figure}[H]
%   \centering
%     \includegraphics[width=\textwidth]{CasoTarzan}
    
%   \caption{}
%   \label{tarzan}
% \end{figure}

% \LaTeX

\end{document}
